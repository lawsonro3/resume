\subsubsection*{\large{Experience}}
\vspace{-0.15in}

% \begin{tabular}{ll}
\textbf{2018--Present: Senior Scientist} \\
Wind and Water Power Program. National Renewable Energy Laboratory. Boulder, CO\\
\vspace{-0.35in}
\begin{itemize}
  \item Worked with a large multi-lab team to develop a blade-resolved wind farm simulation capability using the Nalu CFD code, with the ultimate goal of developing an exascale wind farm CFD modeling tool.
  \vspace{-0.1in}
  \item Supported the DOE Water Power Technologies office in (1) the development of the DOE Marine and Hydrokinetic Program Strategy, (2) in the development of DOE funding opportunity announcements, and (3) in the management of industry projects.
  \vspace{-0.1in}
  \item Studied bat barotrauma from an aerodynamics perspective to help determine the likelihood that barotrauma is a significant contributor to wind turbine-related bat fatalities.
  \vspace{-0.1in}
  \item Studied wind turbine wake propagation under stable atmospheric conditions using the SOWFA CFD tool in order to understand wake physics with the objective of informing simplified wind farm wake models and developing wind farm control strategies.
\end{itemize}
\vspace{-0.1in}
\textbf{2015--2017: Technical Advisor (Management and Operations Contractor)}\\
Wind and Water Power Technologies Office. U.S. Department of Energy. Washington, D.C.\\
\vspace{-0.35in}
\begin{itemize}
  \item Provided technical support for the Department of Energy Wind Program's Atmosphere to Electrons (A2e) program. A2e has the objective of using modern high performance computing and experimental techniques to optimize the performance of wind plants in order to significantly reduce the cost of wind energy.
  \vspace{-0.1in}
  \item Collaborated with Department of Energy staff to develop a U.S. strategy for the development of wave and tidal, ocean, and river current energy technologies.
  \vspace{-0.1in}
  \item Developed funding opportunities to support the deployment of utility-scale wave and tidal/ocean/river current technologies and supported the evaluation of applications.
  \vspace{-0.1in}
  \item Provided technical support for Department of Energy funded National Lab and industry projects.
  \vspace{-0.1in}
  \item Supported the economic analysis and evaluation of wave and tidal/ocean/river current energy technologies.
\end{itemize}
\vspace{-0.1in}
\textbf{2012--2014: Scientist}\\
Water Power Program. National Renewable Energy Laboratory. Boulder, CO\\
\vspace{-0.35in}
\begin{itemize}
  \item PI on the team developing WEC-Sim, an open-source wave energy converter simulation tool that has been widely adopted by industry and academia. The WEC-Sim project was a 4-year NREL-Sandia collaboration. I was the NREL lead developing the WEC-Sim project proposal and once the project was awarded I helped managing an NREL budget of more than \$1M per year and coordinating work between researchers at NREL and Sandia.
  \vspace{-0.1in}
  \item Co-PI on an LDRD project with the objective of developing WEC technologies that have the ability to utilize advanced control strategies to improve performance and reduce loads on wave energy converters. I was the lead on the proposal development team and was responsible for management of the 3-year / \$750K project
  \vspace{-0.1in}
  \item Designed and analyzed a wave energy converters and tidal/ocean current turbines using experimental methods and computational fluid dynamics (CFD) simulations as Part of the DOE Reference Model Project.
  \vspace{-0.1in}
  \item Performed a techno-economic assessment of wave and tidal current energy technologies for the Unites States.
\end{itemize}
\vspace{-0.1in}
\textbf{2010--2012: Postdoctoral Researcher}\\
Water Power Program. National Renewable Energy Laboratory. Boulder, CO\\
\vspace{-0.35in}
\begin{itemize}
  \item Developed a new version of the axial-flow rotor optimization code HARP-Opt.
  \vspace{-0.1in}
  \item Developed computational fluid dynamics (CFD) methods to simulate tidal current turbines.
  \vspace{-0.1in}
  \item Developed a discrete vortex method code to simulate vertical axis turbines.
\end{itemize}
\vspace{-0.1in}
\textbf{2007--2010: National Defense Science and Engineering Graduate Fellow}\\
Gas Dynamics Lab and U.S. Navy Applied Research Lab. Penn State. University Park, PA. \\
\vspace{-0.35in}
\begin{itemize}
  \item Studied the fluid dynamics and odorant transport of canine olfaction.
\end{itemize}
\vspace{-0.1in}
\textbf{2006--2007:  Contractor}\\
Aerodynamics Division. BMW. Munich, Germany\\
\vspace{-0.35in}
\begin{itemize}
  \item Performed vehicle aerodynamics research using CFD techniques.
\end{itemize}
\vspace{-0.1in}
\textbf{2005--2006: Graduate Research Assistant}\\
Experimental and Computational Combustion Lab. Virginia Tech. Blacksburg, VA\\


% \end{tabular}{}

% Department of Energy Wind Power Technologies Office (on assignment from the National Renewable Eneryg Laboratory)
% Washington, DC, 2017 – Present
% Position: Technical Advisor
% • Providing technical support for the Department of Energy Wind Program’s Atmosphere to Electrons (A2e) consortia. A2e has the objective of using modern high performance computing and experimental techniques to optimize the performance of wind plants in order to significantly reduce the cost of wind energy.
% Department of Energy Water Power Technologies Office (on assignment from the National Renewable Eneryg Laboratory)
% Washington, DC, 2015 – 2016
% Position: Technical Advisor
% • Collaborated with Department of Energy staff to develop a U.S. strategy for the devel- opment of wave and tidal/ocean/river current energy technologies
% • Developed funding opportunities to support the deployment of utility-scale wave and tidal/ocean/river current technologies and supported the evaluation of applications
% • Provided technical support for Department of Energy funded National Lab and industry projects
% • Supported the economic analysis and evaluation of wave and tidal/ocean/river current energy technologies
% National Renewable Energy Laboratory
% Boulder, CO, 2012 – 2015
% Position: Senior Research Engineer
% • Lead the development of WEC-Sim, an open-source wave energy converter simulation tool that has been widely adopted by industry and academia
% • Leadateamdevelopingadvancedwaveenergyconversiondevicetechnologiesandcontrol strategies
% • Designed and analyzed a wave energy converters and tidal/ocean current turbines using experimental methods and computational fluid dynamics (CFD) simulations
% • Performed a techno-economic assessment of wave and tidal current energy technologies for the Unites States
% National Renewable Energy Laboratory
% Boulder, CO, 2010 – 2012
% Position: Post-Doctoral Researcher
% • Developed a new version of the axial-flow rotor optimization code HARP-Opt
% • Studied the pressure fields experienced by Bat’s as they fly near wind turbine blades using CFD simulations and used the results to evaluate the theory that Bat’s are dying from barotrauma (i.e. low pressures around the blades)
% • Developed computational fluid dynamics (CFD) methods to simulate tidal current tur- bines
% • Developed a discrete vortex method code to simulate vertical axis turbines
% • Worked with a summer intern to develop a wave energy converter power-take-off model
% in the open-source CFD code OpenFOAM
% Penn State Gas Dynamics Lab and U.S. Navy Applied Research Lab
% University Park, PA, 2007 – 2010
% Position: National Defense Science and Engineering Graduate Fellow
% • Investigated the fluid dynamics and chemical transport phenomena involved in canine olfaction for a project funded by DARPA and the U.S. Transportation Security Labo- ratory
% • Developed a multi-phase CFD model for simulating odorant transport and deposition in the canine nasal airways using the open-source CFD code OpenFOAM
% • Designed and fabricated a model of the canine nasal cavity using rapid prototyping techniques for use in flow visualization experiments
% • Developed seedless particle image velocimetry (PIV) techniques using Schlieren optics for use in compressible flows
% BMW Aerodynamics Division
% Munich, Germany, 2006 – 2007 Position: Graduate Research Assistant
% • Performed vehicle aerodynamics research using CFD techniques
% Virginia Tech Experimental and Computational Combustion Lab
% Blacksburg, VA, 2005 – 2006
% Position: Graduate Research Assistant
% • Developed a method to enhance heat exchanger efficiency using vortex generators using experimental and computational methods
