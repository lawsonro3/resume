%&LaTeX
\documentclass[10pt]{article}

% keep the following two lines; makes very nice pdf files when using ps2pdf
\usepackage[T1]{fontenc}
\usepackage{ae,aecompl}

\usepackage{tabu}
\usepackage{tabularx}
\usepackage[normalem]{ulem}
\usepackage{fullpage}
\usepackage{setspace}
\usepackage{type1cm} % computer modern fonts
\usepackage{fancyhdr}
\usepackage{natbib}
\usepackage{graphicx,psfrag,amsmath,amssymb,subfigure,setspace,rotating}

\usepackage{enumitem}

\usepackage[perpage,para,symbol]{footmisc}

\usepackage[usenames]{color}  % allows colored text

\usepackage[colorlinks]{hyperref}


\setlength{\parindent}{0.0in}
\setlength{\parskip}{0.1in}


\setlength{\headsep}{0.2in}
\setlength{\topmargin}{-0.7in}
\setlength{\evensidemargin}{-0.18in}
\setlength{\oddsidemargin}{0in}
\setlength{\textwidth}{6.7in}
\setlength{\headheight}{0.2in}
\setlength{\textheight}{9.5in}

\pagestyle{empty}

\begin{document}

\begin{centering}
\textbf{Michael J. Lawson}\\
Senior Scientist\\
National Wind Technology Center \\
National Renewable Energy Laboratory\\
\end{centering}


\vspace{-0.0in}
\subsubsection*{Education}
\vspace{-0.1in}

\begin{tabular}{llll}
B.S. & Virginia Tech & Mechanical Engineering & 2005 \\
M.S. & Virginia Tech & Mechanical Engineering & 2006 \\
Ph.D. & The Pennsylvania State University & Mechanical Engineering & 2010 \\
\end{tabular}

% Ph.D. in Mechanical Engineering, The Pennsylvania State University, 2010
% Advisors: Gary Settles and Eric Paterson
% Dissertation: A Fundamental Study of the Airflow and Odorant Transport Phenomena of Canine Olfaction
% M.S. in Mechanical Engineering, Virginia Tech, 2006 Advisor: Dr. Karen Thole
% B.S. in Mechanical Engineering, Virginia Tech, 2005 Minor: Mathematics

\vspace{-0.2in}
\subsubsection*{Experience}

\vspace{-0.1in}
\begin{tabular}{ll}
2018-- & Senior Scientist, Wind and Water Power Program, National Renewable Energy Laboratory\\
2015--2017 & Technical Advisor, Wind and Water Power Technologies Office, U.S. Department of Energy\\
2012--2014 & Scientist, Water Power Program, National Renewable Energy Laboratory\\
2010--2012 & Postdoctoral Researcher, Water Power Program, National National Renewable Energy Laboratory

% Department of Energy Wind Power Technologies Office (on assignment from the National Renewable Eneryg Laboratory)
% Washington, DC, 2017 – Present
% Position: Technical Advisor
% • Providing technical support for the Department of Energy Wind Program’s Atmosphere to Electrons (A2e) consortia. A2e has the objective of using modern high performance computing and experimental techniques to optimize the performance of wind plants in order to significantly reduce the cost of wind energy.
% Department of Energy Water Power Technologies Office (on assignment from the National Renewable Eneryg Laboratory)
% Washington, DC, 2015 – 2016
% Position: Technical Advisor
% • Collaborated with Department of Energy staff to develop a U.S. strategy for the devel- opment of wave and tidal/ocean/river current energy technologies
% • Developed funding opportunities to support the deployment of utility-scale wave and tidal/ocean/river current technologies and supported the evaluation of applications
% • Provided technical support for Department of Energy funded National Lab and industry projects
% • Supported the economic analysis and evaluation of wave and tidal/ocean/river current energy technologies
% National Renewable Energy Laboratory
% Boulder, CO, 2012 – 2015
% Position: Senior Research Engineer
% • Lead the development of WEC-Sim, an open-source wave energy converter simulation tool that has been widely adopted by industry and academia
% • Leadateamdevelopingadvancedwaveenergyconversiondevicetechnologiesandcontrol strategies
% • Designed and analyzed a wave energy converters and tidal/ocean current turbines using experimental methods and computational fluid dynamics (CFD) simulations
% • Performed a techno-economic assessment of wave and tidal current energy technologies for the Unites States
% National Renewable Energy Laboratory
% Boulder, CO, 2010 – 2012
% Position: Post-Doctoral Researcher
% • Developed a new version of the axial-flow rotor optimization code HARP-Opt
% • Studied the pressure fields experienced by Bat’s as they fly near wind turbine blades using CFD simulations and used the results to evaluate the theory that Bat’s are dying from barotrauma (i.e. low pressures around the blades)
% • Developed computational fluid dynamics (CFD) methods to simulate tidal current tur- bines
% • Developed a discrete vortex method code to simulate vertical axis turbines
% • Worked with a summer intern to develop a wave energy converter power-take-off model
% in the open-source CFD code OpenFOAM
% Penn State Gas Dynamics Lab and U.S. Navy Applied Research Lab
% University Park, PA, 2007 – 2010
% Position: National Defense Science and Engineering Graduate Fellow
% • Investigated the fluid dynamics and chemical transport phenomena involved in canine olfaction for a project funded by DARPA and the U.S. Transportation Security Labo- ratory
% • Developed a multi-phase CFD model for simulating odorant transport and deposition in the canine nasal airways using the open-source CFD code OpenFOAM
% • Designed and fabricated a model of the canine nasal cavity using rapid prototyping techniques for use in flow visualization experiments
% • Developed seedless particle image velocimetry (PIV) techniques using Schlieren optics for use in compressible flows
% BMW Aerodynamics Division
% Munich, Germany, 2006 – 2007 Position: Graduate Research Assistant
% • Performed vehicle aerodynamics research using CFD techniques
% Virginia Tech Experimental and Computational Combustion Lab
% Blacksburg, VA, 2005 – 2006
% Position: Graduate Research Assistant
% • Developed a method to enhance heat exchanger efficiency using vortex generators using experimental and computational methods

\end{tabular}

\vspace{-0.2in}
\subsubsection*{Selected Publications}
\vspace{-0.1in}
\begin{enumerate}[leftmargin=1.5pc,itemsep=0pt,parsep=0pt,topsep=0pt,partopsep=1pt]

\item \textbf{Tom N., Lawson, M., Yu, Y., Wright, A.}, 2016. "Spectral Modeling of an Oscillating Surge Wave Energy Converter with Control Surfaces". \emph{Applied Ocean Research}, 56, pp. 143-156.
\item \textbf{Tom N., Lawson, M., Yu, Y., Wright, A.}, 2016. "Development of a Nearshore Oscillating Surge Wave Energy Converter with Variable Geometrys". \emph{Renewable Energy}, 96 (A), pp. 410-424.
\item \textbf{Lawson, M., Barahona Garzon, B., Wendt, F., Yu, Y., Michelen, C.}, 2016. "COER Hydrodynamic Modeling Competition: Modeling the Dynamic Response of a Floating Body Using the WEC-Sim and FAST Simulation Tools". \emph{Proceedings of the ASME 35th International Conference on Ocean, Offshore and Arctic Engineering}, Paper No. OMAE2015-42288.
\item \textbf{V. Neary, M. Previsic, R. Jepsen, M. Lawson, Y. Yu, A. Copping, A. Fontaine, K. Hallett, D. Murray}, 2014. "Methodology for Design and Economic Analysis of Marine Energy Conversion (MEC) Technologies". \emph{SAND2014-9040}
\item \textbf{Lawson, M., Yu, Y., Weber, J., Coe, R., Neary, V.}, 2014. "Extreme Conditions Modeling Workshop Report", \emph{Dept. of Energy Report}, DOE/GO-102014-4450.
\item \textbf{Lawson, M., Craven, B., Paterson, E., and Settles, G.}, 2012. "A Computational Study of Odorant Transport and Deposition in the Canine Nasal Cavity: Implications for Olfaction". \emph{Chemical Senses}, 37 (6), pp. 553-566.
\item \textbf{Hargather, M., Lawson, M., Settles, G., and Weinstein, L.}, 2011. "Seedless Velocimetry Measurements by Schlieren Image Velocimetry". \emph{American Institute of Aeronautics and Astronautics Journal}, 49 (3), pp. 611-620.
\item \textbf{Craven, B., Paterson, E., Settles, G., and Lawson, M.}, 2009. "Development and Verification of a High-fidelity Computational Fluid Dynamics Model of Canine Nasal Airflow". \emph{Journal of Biomechanical Engineering}, 131, pp. 091002.
\item \textbf{Lawson, M., and Thole, K.}, 2008. "Heat Transfer Augmentation Along the Tube Wall of a Louvered Fin Heat Exchanger Using Practical Delta Winglets". \emph{International Journal of Heat and Mass Transfer}, 51(9-10), pp. 2346-2360.

% Journal Publications
% Tom N., Lawson, M., Yu, Y., Wright, A., 2016. “Spectral Modeling of an Oscillating Surge Wave Energy Converter with Control Surfaces”. Applied Ocean Research, 56, pp. 143-156.
% Tom N., Lawson, M., Yu, Y., Wright, A., 2016. “Development of a Nearshore Oscil- lating Surge Wave Energy Converter with Variable Geometrys”. Renewable Energy, 96 (A), pp. 410-424.
% Lawson, M., Craven, B., Paterson, E., and Settles, G., 2012. “A Computational Study of Odorant Transport and Deposition in the Canine Nasal Cavity: Implications for Olfaction”. Chemical Senses, 37 (6), pp. 553-566.
% Lawson, M., Craven, B., Paterson, E., and Settles, G., 2012. “An Experimental Study of Airflow Patterns in an Anatomically-correct Model of the Canine Nasal Cavity”. Experiments in Fluids (In Preparation).
% Hargather, M., Lawson, M., Settles, G., and Weinstein, L., 2011. “Seedless Ve- locimetry Measurements by Schlieren Image Velocimetry”. American Institute of Aeronau- tics and Astronautics Journal, 49 (3), pp. 611-620.
% Craven, B., Paterson, E., Settles, G., and Lawson, M., 2009. “Development and Verification of a High-fidelity Computational Fluid Dynamics Model of Canine Nasal Air- flow”. Journal of Biomechanical Engineering, 131, pp. 091002.
% Lawson, M., and Thole, K., 2008. “Heat Transfer Augmentation Along the Tube Wall of a Louvered Fin Heat Exchanger Using Practical Delta Winglets”. International Journal of Heat and Mass Transfer, 51(9-10), pp. 2346-2360.

% Conference Publications
% Tom N., Yu, Y., Wright, A., Lawson, M., 2016. “Balancing Power Absorption and
% Fatigue Loads in Irregular Waves for an Oscillating Surge Wave Energy Converter”. Proceedings of the ASME 35th International Conference on Ocean, Offshore and Arctic Engineering, Paper No. OMAE2016-55046.
% Quon, E., Platt, A., Yu, Y., Lawson, M., 2016. “Application of the Most Likely Ex- treme Response Method for Wave Energy Convertersl”. Proceedings of the ASME 35th In- ternational Conference on Ocean, Offshore and Arctic Engineering, Paper No. OMAE2016- 54751.
% Lawson, M., Barahona Garzon, B., Wendt, F., Yu, Y., Michelen, C., 2016. “COER Hydrodynamic Modeling Competition: Modeling the Dynamic Response of a Floating Body Using the WEC-Sim and FAST Simulation Tools”. Proceedings of the ASME 35th Interna- tional Conference on Ocean, Offshore and Arctic Engineering, Paper No. OMAE2015-42288.
% Tom N., Lawson, M., Yu, Y., 2015. “Recent Additions in the Modeling Capabilities of an Open-Source Wave Energy Converter Design Tool”. Proceedings of the Twenty-Fifth International Ocean and Polar Engineering Conference, pp. 835-842.
% Yu, Y., Van Rij, J., Coe, R., Lawson, M, 2015. “Preliminary Wave Energy Convert- ers Extreme Load Analysis”. Proceedings of the ASME 34th International Conference on Ocean, Offshore and Arctic Engineering, Paper No. OMAE2015-41532.
% Lawson, M., Yu, Y., Nelessen, A., Ruehl, K., Michelen, C., 2014. “Implementing Nonlinear Buoyancy and Excitation Forces in the WEC-Sim Wave Energy Converter Mod- eling Tool”. Proceedings of the ASME 33rd International Conference on Ocean, Offshore and Arctic Engineering, Paper No. OMAE2014-24445.
% Ruehl, K., Michelen, C., Kanner, S., Lawson, M., Yu, Y., 2014. “Preliminary Verification and Validation of WEC-Sim, an Open-Source Wave Energy Converter Design Tool”. Proceedings of the ASME 33rd International Conference on Ocean, Offshore and Arctic Engineering.
% Lawson, M., Li, Y., and Sale, D., 2011. “Development and Verification of a Computa- tional Fluid Dynamics Model of a Horizontal-axis Tidal Current Turbine”. Proceedings of the 30th International Conference on Ocean, Offshore, and Arctic Engineering, Paper No. OMAE2011-49863.
% Bir, G., Lawson, M., and Li, Y., 2011. “Structural Design of a Horizontal-axis Tidal Current Turbine Composite Blade”. Proceedings of the 30th International Conference on Ocean, Offshore, and Arctic Engineering, Paper No. OMAE2011-50063.
% Hargather, M., Lawson, M., Settles, G., Weinstein, L., and Gogineni, S., 2009. “Focusing-Schlieren PIV Measurements of a Supersonic Turbulent Boundary Layer”. No. AIAA 2009-69, 47th AIAA Aerospace Sciences Meeting Including The New Horizons forum and Aerospace Exposition.
% Lawson, M., Sanders, P., and Thole, K., 2006. “Computational and Experimental Comparison of Tube Wall Heat Transfer Augmented by Winglets in Louvered Fin Heat Exchangers”. No. IMECE2006-14452, ASME 2006 International Mechanical Engineering Congress and Exposition, pp. 681-691.

% Technical Reports
% Yu, Y., Lawson, M., Li, Y., Previsic, M., Epler, J., Lou, J., 2015. “Experimental Wave Tank Test for Reference Model 3 Floating-Point Absorber Wave Energy Converter Project”, Dept. of Energy Report DOE/GO-102014-4450.
% Lawson, M., Yu, Y., Weber, J., Coe, R., Neary, V., 2014. “Extreme Conditions Modeling Workshop Report”, Dept. of Energy Report DOE/GO-102014-4450.
% V. Neary, M. Previsic, R. Jepsen, M. Lawson, Y. Yu, A. Copping, A. Fontaine, K. Hallett, D. Murray (2014). Methodology for Design and Economic Analysis of Marine Energy Conversion (MEC) Technologies. SAND2014-9040
% Musial, W., Lawson. M., Rooney, S., 2013. “Marine and Hydrokinetic Technology (MHK) Instrumentation, Measurement, and Computer Modeling Workshop”, NREL Report NREL/TP-5000-57605.
% Beam, M., Kline, B., Elbing, B., Fontaine A., Lawson M., Thresher, R., and Li, Y, 2012. “Marine Hydrokinetic Turbine Power-Take-Off Design for Optimal Performance and Low Impact on Cost-of-Energy”, NREL Report No. CP-5000-54410..
% Lawson, M., Bir, G., and Thresher, R., 2012. “The Development of a Preliminary De- sign for a Horizontal Axis Tidal Current Turbine”, NREL Report (Under Internal Review).

% Selected Conference Presentations
% Lawson, M., Paterson, E., and Settles, G., 2010. “A Computational Study of Odorant Transport During Canine Olfaction”, U.S. National Committee on Theoretical and Applied Mechanics.
% Lawson, M., Paterson, E., and Settles, G., 2009. “Flow Visualization Experiments In A 4:1 Scale Model of The Canine Nasal Cavity”, American Physics Society - Division of Fluid Dynamics Conference.
% Lawson, M., and Settles, G., 2007. “Focusing-schlieren PIV for the measurement of 3-D turbulent flows”, American Physics Society - Division of Fluid Dynamics Conference.
% Settles, G., Lawson, M., Hargather, M., and Bigger, R., 2007. “Belt-Snap and Towel-snap Shock Waves”, American Physics Society - Division of Fluid Dynamics Confer- ence.
\end{enumerate}

\vspace{-0.2in}
\subsubsection*{Research Interests and Expertise}
\vspace{-0.1in}
\begin{itemize}  \itemsep1pt \parskip0pt \parsep0pt
\item Computational modeling of floating structures and fluid structure interactions
\item Wind farm wake dynamics and wind farm control and performance optimization
\item Design, control, and optimization of wave energy converter systems
\end{itemize}

\vspace{-0.2in}
\subsubsection*{Relevant Experience}
\vspace{-0.1in}
\begin{enumerate}[leftmargin=1.5pc,itemsep=2pt,parsep=0pt,topsep=0pt,partopsep=1pt]
\item \textbf{Code development:} Key member of the team the developed the WEC-Sim wave energy converter design and simulation code (wec-sim.github.io/WEC-Sim/) that is used in the U.S. and internationally to design and analyze the performance wave energy conversion system.
\item \textbf{Management experience:} Supported the management and strategic plan development for U.S. Department of Energy Atmosphere to Electrons (A2e) program that is focused on advancing wind farm technologies through advancing the fundamental physics that govern wind farm performance.
\end{enumerate}
% \vspace{-0.2in}

\end{document}
